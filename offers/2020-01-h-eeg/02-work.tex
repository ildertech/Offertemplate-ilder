\section{Work covered by SLA}
This section will list various named services familiar to GroupM, type of regular work, as well as a service description.

Some of these services are not in our complete control and depend on first party and second party developers within and outside GroupM. We aim for 98 percent or more uptime, but in these cases downtime might be outside our control.

The scope of this agreement is limited to ensuring uptime, fixing of bugs, minor improvements and other small tasks deemed necessary as part of maintenance. This does not include larger pieces of work, such as development of new features.

%feks at me forplikter trond til 30 timer ny-utvikling kvar mnd... (munnleg avtale)
%

\subsection{Named services}


\subsubsection{Access manager}
Access manager is a service used to resolve authorization for internal users within GroupM. 

\textbf{Work description:} Facilitating and maintaining its presence as an app service.

\textbf{Service level delivery:} Requires about 20-26 hours each month in human intervention to keep it at 98 to 99.9 percent uptime.

\subsubsection{API manager}
API manager is a service that exposes most of GroupM's APIs to developers from a single service.

\textbf{Work description:} Surveying and maintaining current APIs as well as adding new APIs.

\textbf{Service level delivery:} Requires about 5-7 hours each month in human intervention to keep it at 98 to 99.9 percent uptime.

\subsubsection{GDPR plugin}
GDPR plugin is not something we currently manage. It is used for GDPR compliance of files sent through email.

\textbf{Work description:} Ensuring the application is running and files are processed properly by the relevant services.

\textbf{Service level delivery:} Requires about 6-10 hours each month in human intervention to keep it at 98 to 99.9 percent uptime. 

\subsubsection{Biztech portal}
The entryway to GroupM's Biztech services on the web.

\textbf{Work description:} Maintaining the service and ensuring integrations such as the chat are working properly in the context of the application.

\textbf{Service level delivery:} Requires about 6-10 hours each month in human intervention to keep it at 98 to 99.9 percent uptime.

\subsubsection{AMC}
Account Manager Cockpit is a tool to compare forecast and actual values on different levels.

\textbf{Work description:} Ensuring the well-being of the application as well as making sure data surfaces as expected.

\textbf{Service level delivery:} Requires about 6-10 hours each month in human intervention to keep it at 98 to 99.9 percent uptime.

\subsubsection{Aeolus}
Automated marketing setup.

\textbf{Work description:} General maintenance and updating of API versions to avoid service interruption.

\textbf{Service level delivery:} Requires about 6-10 hours each month in human intervention to keep it at 98 to 99.9 percent uptime.

\subsection{Variable services}
Besides these named running service, \suppliername provides services at variable numbers of hours depending on \customername needs from month to month beyond the SLA. The rates governing the types of work can be found in \nameref{rates}.

This work needs to be commissioned with reasonable advance notice and a general understanding of varying capacity. For larger projects specific purchase orders needs to be formalized.

\subsubsection{Development}
\textbf{Work description:} Small to medium scale refurbishment, prototyping and new temporary or permantent services.

\textbf{Service level delivery:} On short term purchase orders we can deliver up to 60 hours each month across varying specialites.

\subsubsection{Consulting}
\textbf{Work description:} Senior consulting in UX, design, security and machine learning.

\textbf{Service level delivery:} On short term purchase orders we can deliver up to 30 hours each month.


%Notater
%Deler av API universet - access manager
%Kva har trond?
%%GDPR plugin (trond)
%Liste på 4-5 punkt som er tjenester... ansvarsområder 
%Fornuftig pris på den. 
%Koke ut de tingene som er på ein SLA shortlist.. servicenivå....
%grunnleggande bekymra for accessmanager + API manager... ikkje flytta inn i servicefabric...
%prosjekt å få den ned å vekk.....
%liste ut det me er point of contact