\section{Arbeidspakker}

Denne delen av dokumentet tar for seg arbeid som leder til konkrete leveranser ved avsluttet prosjekt. For å hente informasjon til de ulike arbeidspakkene vil Ilder AS konferere med Hans Petter Torsvik, for å få nødvendige kontakter og samarbeidspartnere for leveransene.

\subsection{Designretningslinjer}
Med bakgrunn i spesifikasjonsdokumentet ønsker vi å komme fram til et førende designfundament som kan lede speifikasjonssarbeidet. Dette vil resultere i fundamentelle retningslinjer for design som gir føring for hvordan systemet skal oppføre seg og oppleves av aktuelle brukere. 

\subsection{Informasjonstekniske utfordringer}
Alle systemer som omfatter persondata i noen grad må designes med utgangspunkt i å beskytte persondata. Ulike føringer for dette må avdekkes og gi rammer for designarbeidet. For eksempel kan vi / bør vi lagre personinformasjon, eller kan vi benytte en tredjepart for persondata, og hvordan dette gir føringer for hvordan brukeren opplever systemet.

\subsection{Tekniske anbefalinger}
Ilder AS vil se helhetlig på spesifikasjonen og med dette anbefale en teknisk løsning for å realisere spesifikasjonen vi kommer frem til. Her vil vi anbefale løsninger for webteknologi, app-teknologi og skyløsninger som gir mening.

\subsection{Brukerreiser}
Detaljere brukerreiser fundamentert i designretningslinjene fra ulike perspektiv som turist, reiseoperatør, opplevelsetilbyder, administrator etc. Her er fokuset å finne ut hvordan brukere ønsker å oppleve et slikt system og hvordan de vil oppleve å være en del av et slikt system.

\subsection{Mockups}
Illustrere med wireframes/mockups hvordan to til tre brukerreiser gjennom systemet utarter seg. Vanlige oppgaver en tar for seg i systemet som for eksempel å legge inn en bestilling, legge inn bestilte flybilletter, få opp varsel om en utsettelse etc. Hvem er involvert? Hvordan er de involvert?

\subsection{Utvalg av brukerhistorier}
Det vil ikke være hensiktsmessig å detaljere alle brukerhistorier i denne omgang, men heller et utvalg som illustrerer hovedsaklig funksjonalitet.

\subsection{Estimering og prosjektering}
Dokumentet vil foreslå en fremdriftsplan, samt et overordnet budsjett og tidslinje mot et minimalt levedyktig produkt som kan bygges videre på.

\section{Leveranser}
Alle leveransar blir overlevert på engelsk for å kunne kommunisere med utviklere og designere både nasjonalt og internasjonalt, under og i etterkant av leveransen.

\subsection{Dokumentasjon og design}
Dokumentasjon som inneholder på engelsk:

\begin{center}
\begin{tabu} to 1.0\textwidth {    X[l]  }
 \hline
Design guidelines \\
Selected user journeys \\
Selected user stories \\
Technical challenges \\
Technical recommendations \\
Project estimations towards and MVP \\
Selected low fidelity mockups \\
Selected high-fidelity mockups \\
 \hline
\end{tabu}
\end{center}

Det blir overlevert  mockups via Zepelin.io eller lignende, samt videogjennomgang av wireframes.

\subsection{Administrering}
Vi foreslår 3-4 møter underveis for status, eventuelle endringer og forventingsjustering med \customername. Ilder AS administrerer selv dialogen med brukere.

\subsection{Tidslinje}

\begin{center}
\begin{tabu} to 0.95\textwidth {  X[l]  X[l]  }
 \hline
 Oktober 2019 & Tilbud akseptert \\ 
 + 1 uke & Oppstart og datafangst  \\
 + 1 uke & Brukerinvolvering, første skisser   \\
 + 2 uke & Ferdigstilling skisser \\
 + 2 uke & Ferdigstilling dokumentasjon og overlevering\\
 \hline
\end{tabu}
\end{center}





