\documentclass{article}

\usepackage[utf8]{inputenc}
\usepackage{graphicx}
\usepackage{hyperref}
\usepackage{epigraph}
\usepackage{float}
\usepackage[usenames, dvipsnames]{color}
\usepackage{parskip}
\usepackage{svg}
\usepackage{array}
\usepackage{tabu}
\usepackage{tikz}
\usepackage{enumitem}
\usepackage{amsfonts}

\newcommand{\amount}{6in}   %%<---- adjust
\newcommand\tab[1][0.8cm]{\hspace*{#1}}

%checkmarks
\setlist[itemize,2]{label=$\checkmark$}


% Litt luft i margen for notater
\usepackage[margin=1.5in]{geometry}
\renewcommand*\familydefault{\sfdefault}

% Fin spacing mellom linjene
\linespread{1.5}

%\usepackage{draftwatermark}
%\SetWatermarkText{DRAFT}
%\SetWatermarkScale{3}

\begin{document}



\newcommand{\suppliername}{Ilder AS }
\newcommand{\supplieradress}{Brugata 6, 5200 Os }

\newcommand{\customername}{Norway Insight}
\newcommand{\customeradress}{Holmedalsgården 3, 2 Floor. 5003 Bergen}

\newcommand{\commencedate}{2019-10-07}




\definecolor{marinecolor}{RGB}{51,51,51}




%rgb 51,51,102


\vspace{2.5in}
\begingroup
\centering
\par\normalfont\fontsize{15}{20}\sffamily\selectfont
\textbf{Kravspesifikasjon for Norway Insight}\\
{\LARGE }\par % Book title
\vspace*{1cm}
{\large{av} }\par 
\vspace{2cm} 
\begin{center}
  \centering
  \includesvg{images/logo-small-solo.svg}
\end{center}
{\small{\commencedate} }\par % Author name
\endgroup
%Tweaking epigraph
\setlength\epigraphwidth{.8\textwidth}
\renewcommand{\epigraphflush}{center}


\newpage
\section{Introduksjon}
Dette tilbudet kommer på forespørsel fra Hans Petter Torsvik, avtalt i videomøte 1. oktober 2019, og basert på tilsendt presentasjon av prosjektet til Norway Insight.

\subsection{Bakgrunn for tilbud}
Presentasjonen og møtet ga grunnlag for videre arbeid med kravspesifisering. Dette grunnlaget legger føringar for en del ønskede leveranser \customername har behov for med å kunne komme videre i prosessen med realisering av en digital reisehub. 

Beskrivelse: \break
Norway Insights gründer har gjennom 14 år bygget opp et unikt nettverk og en innsikt innen bærekraftig opplevelsesbasert turisme på Vestlandet. Denne innsikten gir oss mulighet til å skape en unik kundeopplevelse i 3 dimensjoner/faser;

1.     Planleggings- og «drømme»-fasen – dette skal være en opplevelse i seg selv som man ønsker å dele med andre.

2.     Gjennomførings-/leveransefasen – her legger vi listen høyt og har som mål å overraske den reisende med opplevelser «on the fly» når kunden først er i et område (f.eks. være med å sanke egg på den lokale gården)

3.     Minne- og delingsfasen, her skal vi sikre god rating og at minnene og opplevelsene lever videre og deles på sosiale media. Vi skal også kunne fylle på med oppdatert informasjon om stedene man har besøkt.

Den unike kundeopplevelsen skal være vårt fortrinn som skal gi økt kundevolum og økt interesse og engasjement hos våre underleverandører.

 

\newpage
\section{Arbeidspakker}

Denne delen av dokumentet tar for seg arbeid som leder til konkrete leveranser ved avsluttet prosjekt. For å hente informasjon til de ulike arbeidspakkene vil Ilder AS konferere med Hans Petter Torsvik, for å få nødvendige kontakter og samarbeidspartnere for leveransene.

\subsection{Designretningslinjer}
Med bakgrunn i spesifikasjonsdokumentet ønsker vi å komme fram til et førende designfundament som kan lede speifikasjonssarbeidet. Dette vil resultere i fundamentelle retningslinjer for design som gir føring for hvordan systemet skal oppføre seg og oppleves av aktuelle brukere. 

\subsection{Informasjonstekniske utfordringer}
Alle systemer som omfatter persondata i noen grad må designes med utgangspunkt i å beskytte persondata. Ulike føringer for dette må avdekkes og gi rammer for designarbeidet. FOr eksempel kan vi / bør vi lagre personinformasjon, eller kan vi benytte en tredjepart for persondata, og hvordan dette gir føringer for hvordan brukeren opplever systemet.

\subsection{Tekniske anbefalinger}
Ilder AS vil se helhetlig på spesifikasjonen og med dette anbefale en teknisk løsning for å realisere spesifikasjonen vi kommer frem til. Her vil vi anbefale løsninger for webteknologi, app-teknologi og skyløsninger som gir mening.

\subsection{Brukerreiser}
Detaljere brukerreiser fundamentert i designretningslinjene fra ulike perspektiv som turist, reiseoperatør, opplevelsetilbyder, administrator etc. Her er fokuset å finne ut hvordan brukere ønsker å oppleve et slikt system og hvordan de vil oppleve å være en del av et slikt system.

\subsection{Mockups}
Illustrere med wireframes/mockups hvordan to til tre brukerreiser gjennom systemet utarter seg. Vanlige oppgaver en tar for seg i systemet som for eksempel å legge inn en bestilling, legge inn bestilte flybilletter, få opp varsel om en utsettelse etc. Hvem er involvert? Hvordan er de involvert?

\subsection{Utvalg av brukerhistorier}
Det vil ikke være hensiktsmessig å detaljere alle brukerhistorier i denne omgang, men heller et utvalg som illustrerer hovedsaklig funksjonalitet.

\subsection{Estimering og prosjektering}
Dokumentet vil foreslå en fremdriftsplan, samt et overordnet budsjett og tidslinje mot et minimalt levedyktig produkt som kan bygges videre på.

\section{Leveranser}
Alle leveransar blir overlevert på engelsk for å kunne kommunisere med utviklere og designere både nasjonalt og internasjonalt, under og i etterkant av leveransen.

\subsection{Dokumentasjon og design}
Dokumentasjon som inneholder på engelsk:

\begin{center}
\begin{tabu} to 1.0\textwidth {    X[l]  }
 \hline
Design guidelines \\
Selected user journeys \\
Selected user stories \\
Technical challenges \\
Technical recommendations \\
Project estimations towards and MVP \\
Selected low fidelity mockups \\
Selected high-fidelity mockups \\
 \hline
\end{tabu}
\end{center}

Det blir overlevert  mockups via Zepelin.io eller lignende, samt videogjennomgang av wireframes.

\subsection{Administrering}
Vi foreslår 3-4 møter underveis for status, eventuelle endringer og forventingsjustering med \customername. Ilder AS administrerer selv dialogen med brukere.

\subsection{Tidslinje}

\begin{center}
\begin{tabu} to 0.95\textwidth {  X[l]  X[l]  }
 \hline
 Oktober 2019 & Tilbud akseptert \\ 
 + 1 uke & Oppstart og datafangst  \\
 + 1 uke & Brukerinvolvering, første skisser   \\
 + 2 uke & Ferdigstilling skisser \\
 + 2 uke & Ferdigstilling dokumentasjon og overlevering\\
 \hline
\end{tabu}
\end{center}







\section{Priser}
Dette er et fastpristilbud begrenset oppad til 80 timer og 6 uker.

\begin{center}
\begin{tabu} to 0.95\textwidth {  X[l]  X[r]   X[r]  }
 \textbf{Fastpris}  & \textbf{Eks MVA}  & \textbf{Inkl. MVA}\\

 \hline
    Leveranse  & 96 250 ,-  & 120 313,-\\
  \hline
     &  \textbf{Totalt} &  \textbf{120 313,-} \\ 
\hline
\end{tabu}

\end{center}


\subsection{Betaling}
50 \% faktureres ved skriftlig aksept av tilbud, med 14 dagers betalingsfrist. 50 \% faktureres ved overlevering av dokumentasjon og design.

\newpage
\section{Signaturer}





\begin{center}
For leverandør - \suppliername
\vspace{1cm}

\begin{tabu} to 0.75\textwidth {  X[l]  X[l]  }

 Signatur & \hrulefill  \\ 
 Navn & Jon Arild Jacobsen  \\ 
 Stilling & Daglig leder  \\ 
 Dato & 2019-10-08  \\ 

\end{tabu}
\end{center}

\vspace{2cm}

%todo ordne opp i signaturgreiene, slik at det blir pretty lines

\begin{center}
For kunde - \customername	
\vspace{1cm}

 \begin{tabu} to 0.75\textwidth {  X[l]  X[l]  }
 Signatur &  \hrulefill \\ 
 Navn & \hrulefill  \\ 
 Stilling & \hrulefill  \\ 
 Dato & \hrulefill  \\ 
 \end{tabu}

\end{center}


\end{document}
